%% Простая презентация с примером включения программного кода и
%% пошаговых спецэффектов
\documentclass{beamer}
\usetheme{SPbAU}
%\useoutertheme{infolines}
\usepackage{fontspec}
\usepackage{xunicode}
\usepackage{xltxtra}
\usepackage{xecyr}
\usepackage{hyperref}
\setmainfont[Mapping=tex-text]{DejaVu Serif}
\setsansfont[Mapping=tex-text]{DejaVu Sans}
\setmonofont[Mapping=tex-text]{DejaVu Sans Mono}
\usepackage{polyglossia}
\usepackage{csquotes}
\setdefaultlanguage{russian}
\usepackage{graphicx}
\usepackage{listings}
\usepackage{xpatch}
\lstdefinestyle{highlight}{
  basicstyle=\footnotesize\ttfamily\color{black},
  keywordstyle=\bfseries\color{black},
  commentstyle=\itshape\color{black},
}
\lstdefinestyle{base}{
  belowcaptionskip=1\baselineskip,
  breaklines=true,
  xleftmargin=\parindent,
  showstringspaces=false,
  basicstyle=\footnotesize\ttfamily\color{black!40},
  keywordstyle=\bfseries\color{black!40},
  commentstyle=\ttfamily\itshape\color{black!40},
  stringstyle=\color{red},
  numbers=left,
  numbersep=5pt,
  numberstyle=\tiny\color{gray},
  morecomment=[s]{(*}{*)},
  texcl=true,
  literate={п}{\cyrp}1
             {и}{\cyri}1,
  moredelim=**[is][\only<+>{\color{black}\lstset{style=highlight}}]{§}{§},
  morekeywords={implicit, module, struct, sig, val}
}
\lstdefinestyle{base2}{
  belowcaptionskip=1\baselineskip,
  breaklines=true,
  xleftmargin=\parindent,
  showstringspaces=false,
  basicstyle=\footnotesize\ttfamily\color{black},
  keywordstyle=\bfseries\color{black},
  commentstyle=\ttfamily\itshape\color{black},
  stringstyle=\color{red},
  numbers=left,
  numbersep=5pt,
  numberstyle=\tiny\color{gray},
  morecomment=[l]{(*},
  texcl=true,
  literate={п}{\cyrp}1
             {и}{\cyri}1,
  moredelim=**[is][\only<+>{\color{black}\lstset{style=highlight}}]{§}{§},
  morekeywords={implicit, module, struct, sig, val}
}
\newcommand{\backupbegin}{
   \newcounter{framenumberappendix}
   \setcounter{framenumberappendix}{\value{framenumber}}
}
\newcommand{\backupend}{
   \addtocounter{framenumberappendix}{-\value{framenumber}}
   \addtocounter{framenumber}{\value{framenumberappendix}} 
}

\usepackage[backend=biber, style=authortitle, doi=false, url=false, isbn=false]{biblatex}
\xapptobibmacro{cite}{\setunit{\nametitledelim}\printfield{year}}{}{}
\addbibresource{presentation-short.bib}

\begin{document}
\title[Стратегии редукции термов]{Эвристические стратегии редукции термов в задачах формальной верификации}
\author[Трилис А.А.]{Трилис Алексей Андреевич\\{\footnotesize\textcolor{gray}{научный руководитель: А.М.Ляшин}}}
\institute{НИУ ВШЭ --- Санкт-Петербург}
\date{?? июня 2025 г.}
\frame{\titlepage}

\begin{frame}\frametitle{Введение}
\begin{itemize}
  \item TODO
  \item TODO
\end{itemize}
\end{frame}

\lstset{language=haskell}
\begin{frame}\frametitle{Обзор литературы}
\begin{itemize}
  \item Согласованность~--- каждая типизация программы должна приводить к одной и той же семантике
  \item Каноничность~--- в области видимости не более одного экземпляра для каждого типа 
  \item Решения с каноничностью
  \begin{itemize}
    \item Haskell, Rust
    \item В ML невозможно проверить каноничность
    \item Есть решение для ML \footcite{ml_typeclasses}, но оно вводит серьёзные ограничения
  \end{itemize}
  \item Решения без каноничности
  \begin{itemize}
    \item Scala
    \item Решение для OCaml \footcite{white} в основной язык интегрировать пока не получилось
    \item Система проверки типов недостаточно сильна
  \end{itemize}
\end{itemize}
\end{frame}

\begin{frame}\frametitle{1}
\end{frame}

\begin{frame}\frametitle{2}
\begin{itemize}
  \item TODO
  \item TODO
\end{itemize}
\end{frame}

\begin{frame}\frametitle{Цель и задачи}
\textbf{Цель}: TODO

\textbf{Задачи}:
\begin{itemize}
  \item TODO
  \item TODO
  \item TODO
  \item TODO
\end{itemize}
\end{frame}

\begin{frame}\frametitle{3}
\begin{itemize}
  \item TODO
  \item TODO
\end{itemize}
\end{frame}

\begin{frame}\frametitle{4}
\begin{itemize}
  \item TODO
  \item TODO
\end{itemize}
\end{frame}

\begin{frame}\frametitle{5}
\begin{itemize}
  \item TODO
  \item TODO
\end{itemize}
\end{frame}

\begin{frame}\frametitle{6}
\begin{itemize}
  \item TODO
  \item TODO
\end{itemize}
\end{frame}

\begin{frame}\frametitle{7}
TODO
\end{frame}

\begin{frame}\frametitle{8}
\begin{itemize}
  \item TODO
  \item TODO
\end{itemize}
\end{frame}

\begin{frame}\frametitle{Результаты}
\begin{itemize}
    \item TODO
    \item TODO
    \item TODO
    \item TODO
\end{itemize}
\end{frame}

\appendix
\backupbegin

\begin{frame}\frametitle{Apx 1}
TODO
\end{frame}

\begin{frame}\frametitle{Apx 2}
TODO
\end{frame}

\begin{frame}\frametitle{Apx 3}
\begin{itemize}
  \item TODO
  \item TODO
\end{itemize}
\end{frame}

\begin{frame}\frametitle{Apx 4}
TODO
\end{frame}


\begin{frame}\frametitle{Apx 5}
\begin{itemize}
  \item TODO
  \item TODO
\end{itemize}
\end{frame}


\backupend
\end{document}
