\documentclass[../diploma.tex]{subfiles}
\begin{document}\label{sec:3}

Данная глава посвящена сравнению описанных выше стратегий. В ней описывается подготовка необходимых тестовых данных, приведены результаты экспериментов с применением стратегий на этих данных, и сделаны соответствующие выводы.

\subsection{Подготовка тестовых данных}

Тестовые данные для этой работы состоят из двух частей. Первая, синтетическая, была написана специально для этой работы и представляет из себя несколько наборов функций, в каждом из которой размер и сложность функций равномерно увеличивается по заданному шаблону. Вторая часть взята из реальной практики верификации смарт-контрактов.

Для того, чтобы синтетические данные подходили для численных экспериментов, необходимо было создать логически несложный код, который может повторяться с незначительными изменениями неограниченное число раз. Вместе с этим необходимо было протестировать такие фундаментальные элементы языков программирования, как условные переходы и рекурсия. В то же время циклы не были включены в этот код, поскольку верификация кода с циклами является отдельной областью исследований, ставящих множество уникальных вопросов, и выходит за рамки этой работы. [сослаться что циклы это сложно?]

Для построения синтетических тестовых данных был выбран классический алгоритм подсчёта полиномиального хеша строки. Этот алгоритм был реализован на языке Ursus в различных вариациях. Синтетический набор состоит из следующих наборов функций:

\begin{itemize}
    \item \textbf{Simple}. $i$-я функция вычисляет хеш первых $i$ символов строки, при этом код является линейным (не использует циклы или рекурсию), то есть размер кода линейно зависит от порядкового номера функции.
    \item \textbf{If}. Расширение набора \textbf{Simple}, в котором симулируется работа с нуль-\\ терминированными строками. В случае, если программа дошла до нулевого символа, вычисление должно завершиться. Этот набор позволяет протестировать условный оператор и досрочный выход из функции.
    \item \textbf{Recursion}. Рекурсивная версия набора \textbf{Simple}. В ней функция с номером $i+1$ сначала вызывает функцию $i$, а затем дополняет вычисление $(i+1)$-м символом.
    \item \textbf{IfAndRecursion}. Рекурсивная версия набора \textbf{If}.
\end{itemize}

Для реальной части тестовых данных был выбран кошелек с мультиподписью из экосистемы TON [сослаться на код]. Кошельки с мультиподписью используются в блокчейн-сетях повсеместно[?] и являются критическим элементом для безопасности многих систем, поэтому верификация таких кошельков имеет практическую значимость. Материалы для верификации этого смарт-контракта были взяты из практики компании Pruvendo, в том числе трансляция кода с исходного языка Solidity на Ursus и формальная спецификация этой системы. Эксперименты заключались в применении разных стратегий для доказательства имеющейся формальной спецификации.

Все эксперименты поставлены на процессоре ??? с ??? оперативной памяти и ??? ядрами.

\subsection{2}

TODO

\subsection{Выводы и результаты по главе}

TODO

\end{document}
