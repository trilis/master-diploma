\documentclass[../thesis.tex]{subfiles}
\begin{document} \label{sec:introduction}

Формальная верификация является важной и активно развивающейся областью исследований. В программных продуктах нередко встречаются ошибки, которые трудно заметить даже опытному программисту. Традиционные методы нахождения программных ошибок, такие как тестирование, хоть и зачастую являются эффективными, не могут гарантировать полное отсутствие ошибок, так как проверка происходит лишь на ограниченном множестве входных данных. В то же время в некоторых областях, таких как авиация, медицина, финансы, цена ошибки может оказаться крайне высокой, и уровень корректности, обеспечиваемый тестированием, перестаёт являться приемлемым. В таких областях требуется математически доказать корректность системы, то есть её соответствие формальной спецификации.

Интерактивное средство доказательств теорем Coq, также известное как Rocq \cite{coq} используется во многих работах по математике и формальной верификации. В частности, знаменитая теорема о четырёх красках была доказана именно в этой системе \cite{4colors}, а среди крупных проектов по формальной верификации на Coq можно выделить доказанно корректный оптимизирующий компилятор для подмножества языка C CompCert \cite{compcert} и фреймворк для верификации распределённых систем IronFleet \cite{ironfleet}.

Coq предоставляет мощный и выразительный язык для формальной верификации. Этот язык содержит богатую систему типов, предлагает удобные возможности для расширения, в том числе разработку стратегий доказательства на языке Ltac. Вместе с этим, производительность языка может вызывать практические проблемы. На некоторых практических задачах вычисления могут показывать экспоненциальное время работы \cite{coqslow} \cite{gross_phd}.

Смарт-контракты, то есть программы, запускаемые в рамках блокчейн-сетей —-- особенно подходящее применение для формальных методов \cite{smart_contracts}, поскольку эти программы, с одной стороны, являются относительно простыми, а с другой зачастую оперируют большими денежными суммами, то есть цена ошибки может быть высокой. Вдобавок, изменение таких программ с целью исправления ошибок после их запуска в силу особенностей блокчейн-сетей является сложным или даже невозможным, поэтому необходимо найти ошибки до запуска. Язык Ursus \cite{ursus}, улучшению которого посвящена эта работа, использует возможности расширения Coq для создания системы верификации смарт-контрактов. 

В рамках проекта Ursus были достигнуты значительные результаты по поддержке языков программирования, используемых в разработке смарт-контрактов, таких как Solidity и Rust, и по автоматизации процесса верификации, от трансляции исходного кода во внутренний язык Ursus до непосредственно доказательства свойств кода. Однако, поскольку Ursus расширяет и использует средства Coq, во многих практических задачах непреодолимым препятствием становятся вышеобозначенные проблемы Coq с производительностью.

При всех возможностях расширения Coq, порядки вычисления и унификации, приводящие к экспоненциальным проблемам, остаются "чёрным ящиком", и возможности пользователя повлиять на эти алгоритмы крайне ограничены. Даже нахождение причины экспоненциального поведения может потребовать глубокого исследования внутренних алгоритмов Coq \cite{gross_phd}. Поэтому актуальной является задача по разработке более гибких стратегий, которые, используя нативные методы вычисления в Coq для небольших элементов, будут композироваться в итоговое вычисление специфичными для стратегии методами, позволяя тем самым пользователю влиять на производительность посредством выбора подходящей стратегии.

Центральной гипотезой этой работы заключается в том, что разработанные таким способом стратегии вычислений будут по крайней мере в некоторых случаях более производительными, чем нативные стратегии Coq. Эта работа фокусируется на практической задаче оптимизации для конкретного процесса в системе Ursus, на символьном вычислении результата функции относительно необходимого свойства. Этот процесс является центральным для доказательств --- по окончании работы этого процесса, верификатор может работать с минимализированными формулами и утверждениями, относящимся непосредственно к доказываемому свойству программы, что качественно более удобно, чем доказательства с нуля. Вместе с этим, символьное вычисление на данный момент является самым вычислительно затратным процессом в Ursus. Таким образом, оптимизация этого процесса является ключевой практической задачей для системы Ursus.

Отдельный интерес представляет исследование эвристик, позволяющих тонкой настройкой улучшить производительность стратегий для повторяющихся закономерностей во входных данных для конкретной практической задачи. В этой работе будут исследованы эвристики, связанные с типами и с графовыми характеристиками данных.

\subsection*{Цель и задачи}

Целью данной работы является оптимизация символьного вычисления результата функции в контексте системы Ursus. Для этого ставятся следующие задачи:

\begin{itemize}
	\item Разработать несколько стратегий вычисления, используя классические порядки редукций и эвристики, продиктованные структурой специфических для системы Ursus данных.
	\item Подготовить набор программ на Ursus, включающий в себя фундаментальные конструкции языков программирования, для использования в качестве тестовых данных.
	\item Сравнить разработанные стратегии на тестовом наборе и выявить среди них наиболее производительные.
\end{itemize}

\subsection*{Обзор последующих глав}

Глава \ref{sec:1} посвящена обзору теоретических основ, относящихся к области исследования, а также подробному описанию контекста работы.

В главе \ref{sec:2} описываются предложенные стратегии вычисления с различными эвристическими модификациями.

Глава \ref{sec:3} содержит описание тестового набора программ и сравнение предложенных стратегий с точки зрения производительности.

\end{document}