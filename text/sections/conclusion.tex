
\documentclass[../thesis.tex]{subfiles}
\begin{document}

В рамках данной работы были достигнуты следующие результаты: 

\begin{itemize}
    \item Разработаны стратегии вычисления, всего 40 различных вариаций. В том числе, представлена методология разработки эвристических оптимизаций, основанных на структурных свойствах данных. 
    \item Для сравнения стратегий создан набор программ на языке Ursus, включающий фундаментальные конструкции языков программирования.
	\item Все приведенные в тексте работе стратегии реализованы на языке Ltac. Эта реализация и тестовые наборы доступны в репозитории \url{https://github.com/trilis/ursus-solver-strategies}
    \item По результатам сравнения стратегий, лучшая из них на некоторых программах (с большой глубиной рекурсии и малым числом условных операторов) демонстрируют асимптотическое улучшение в производительности относительно базовой, а именно полиномиальный рост времени работы относительно экспоненциального.
    \item Несмотря на то, что на некоторых программах улучшение производительности менее выражено, самая эффективная стратегия на всех рассмотренных программах не показывает результат значимо хуже любой другой стратегии. Это позволяет рекомендовать её использование во всех случаях, без предварительного анализа характеристик программы.
    \item Наиболее эффективная стратегия будет использоваться в дальнейших коммерческих проектах по формальной верификации, поскольку оптимизации имеют практическое значение для развития системы Ursus.
\end{itemize}

\end{document}